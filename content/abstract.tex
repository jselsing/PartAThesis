% !TEX root = Clean-Thesis.tex
%
\pdfbookmark[0]{Abstract}{Abstract}
\chapter*{Abstract}
\label{sec:abstract}
\vspace*{-10mm}

Because of the vast expanse of the universe we live in, very little starlight reaches us from the environments furthest from us. To explore the region of space unavailable to us via direct detection, extremely energetic phenomena can be used to investigate the space between us and them. By gaining a thorough understanding of the physical processes at play in the high-energy events populating our universe,  any signatures deviating from the intrinsic appearance of the object can be attributed to the intervening material and thus an indirect inference can be made about the dark universe. In this thesis I present some of the work I am involved with related to high-energy phenomena, specifically: Quasars, Gamma-Ray Bursts and Supernovae. For quasars (QSOs), I have researched the average properties of a sample of bright $M_{r} \leq -27.5$ QSOs and created a composite spectrum for community usage. I showed a application of the composite in inferring dust content of the host galaxies. Additionally I find a steeper slope of the quasar power-law continuum as compared to the traditionally assumed one, indicating an intrisically harder continuum. For GRBs, I am involved with the X-shooter GRB collaboration and am investigating the average properties of the GRB optical afterglows for the sample we are building. Data-collection is continuing and I prepare the normalized afterglows for the collaboration. Once a sufficient sample size is reached, I will lead a project to construct a composite afterglow spectrum. 
