% !TEX root = Clean-Thesis.tex
%
\pdfbookmark[0]{Abstract}{Abstract}
\chapter*{Abstract}
\label{sec:abstract}
\vspace*{-10mm}

Because of the vast expanse of the universe we live in, very little starlight
reaches us from the environments furthest from us. To explore the region of
space unavailable to us via direct detection, extremely energetic phenomena can
be used to investigate the space between us and them, by illuminating the
universe otherwise obscured. Gaining a thorough understanding of the physical
processes at play in the high-energy events populating our universe is of prime
importance because any signatures deviating from the intrinsic appearance of the
object can be attributed to the intervening material and thus an indirect
inference can be made about the dark universe. In this thesis I present some of
the work I am involved with related to high-energy phenomena, specifically:
Quasars (QSOs), Gamma-Ray Bursts (GRBs) and Supernovae (SNe). The work I present
here consists of four projects I am involved with and the majority is still a
work in progress. The work I have been doing is specifically on the
\textit{average} properties of the UV to near-infrared spectra of a selection of
QSOs, the explosion environments of high-velocity ejecta core-collapse
supernovae, the average optical afterglow of long-duration GRBs and the search
for high-redshift, lensed SNe, including the quadruply lensed supernova, "SN
Refsdal".




