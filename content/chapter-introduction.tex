% !TEX root = ../thesis-example.tex
%
\chapter{Introduction}
\label{sec:intro}

\cleanchapterquote{Astronomy? Impossible to understand and madness to investigate.}{Sophocles}{}

In this Part A Thesis I will summarize some of the work I have been doing the first two years of my PhD, while also officially enrolled in the masters programme. 


For quasars (QSOs), I have researched the average properties of a sample of bright $M_{r} \leq -27.5$ QSOs and created a composite spectrum for community usage. I showed an application of the composite in inferring dust content of the quasar host galaxies. Additionally, for the composite I find a steeper slope of the quasar power-law continuum as compared to the traditionally assumed one, indicating an intrinsically harder continuum. 
For GRBs, I am involved with the X-shooter GRB collaboration and am investigating the average properties of the GRB optical afterglows for the sample we are building. Data-collection is continuing and I have developed an algorithm to normalize the afterglows which is then used for the emission line studies. Once a sufficient sample size is reached, I will lead a project to construct a composite afterglow spectrum. 
For Supernovae, I am involved in two projects. One related to the explosion environments of type Ic and Ic-BL supernovae, where we have obtained a sample of 19 supernova host galaxies observed with an IFU, allowing spatially resolved diagnostics to be calculated for the stellar progenitor population, to investigate whether the explosion environments of the two types of supernovae differ. The other supernova-project is a collaboration with the Frontier Fields program, where spectroscopic follow-up of potential high-redshift supernovae are carried out using X-shooter. As part of this work I am working on the quadruply lensed supernova "SN Refsdal" where the X-shooter observations have helped constrain the type.


\section{Quasars: Cosmic Lighthouses}
\label{sec:intro:qso}



\section{Gamma-Ray Bursts: Flashes in the Dark}
\label{sec:intro:vimos}


\section{Supernovae: The fiery deaths of massive stars}
\label{sec:intro:sn}

Some of nature's most powerful explosions, long
  GRBs and different kinds of core-collapse SN, are linked to the
  deaths of massive stars. However, neither the progenitor system nor
  the production conditions that lead to each kind of explosion in a
  massive stripped star are well understood. The different kinds of
  Supernovae (SNe) are determined by the initial mass and metallicity
  of the stellar progenitor, as well as by the metallicity-dependent
  mass loss in the stellar winds at the end phase of their evolution
  and the interaction with a sufficiently close companion star. Type
  Ic SNe (SNe Ic) are explosions from the most heavily stripped
  massive stars that have been removed of both their H and He
  layers. Long duration gamma-ray bursts (GRBs) may be the most
  extreme cases of stellar explosions, and a few cases have been
  associated with broad-lined Type Ic SN \citep{Woosley2006}, while other long GRBs
  surprisingly lacked the distinct SN signatures \citep{Fynbo2006}. The connection
  between long-duration GRBs and broad-lined SNe Ic (SNe
  Ic-bl/hypernovae) and the existence of SNe Ic-bl without observed
  GRBs raises the question of what distinguishes a GRB progenitor from
  that of an ordinary SN Ic-bl without a GRB. This question may be
  answered by observing the stellar progenitors, which give rise to
  the various types of SN explosions. However, searches for SN
  progenitors in images taken before the explosions have failed,
  because the galaxies are distant and individual stars are difficult
  to distinguish \citep{Maund2005}.

\hspace{0.4cm} An indirect method is to explore the environments at
the locations of the SN and GRBs to look for systematic trends. SN
Ic-bl without observed GRBs lie in systematically more metal rich
environments than SNe with GRBs \citep{Modjaz2008}. Integral field data of the host
of GRB\,980425/SN\,1998bw show that the location of the SN (a Type
Ic-bl) is close to a very metal poor region, but that the metallicity
variations in HII regions in the dwarf galaxy host are otherwise minor
\citep{Christensen2008}.  In comparison Fig. 1 shows the Type Ic sites to be relative
metal rich. Recently the abundances at the sites of different kinds of
core collapse SNe have been determined and shows that SNe Type Ic lie
in systematically more metal rich environments than other types of
core collapse explosions (Fig. 3) \citep{Modjaz2011, Leloudas2011, Kuncarayakti2013a, Kelly2012}.  In contrast \citet{Anderson2010} argue
that metallicity is not the dominant parameter for the SN type. The
metallicity difference may have a physical origin, as a higher
metallicity may give rise to stronger line-driven winds that remove
the outer H and He layers of the star \cite{Vink2005}, a necessary condition for
a Type Ic SNe. This argument does not explain the lower abundances at
the sites of SNe Ic-bl as well as at GRB regions, since GRB
progenitors are expected to be metal poor \citep{Woosley1993}, unless the GRB-SN
progenitors underwent a peculiar evolutionary phase marked by chemical
homogeneous mixing. %A possible solution to this dilemma is a massive
%binary progenitor system, rather than a metallicity effect.

\hspace{0.4cm}
Using integrated central oxygen abundance of the host
galaxy as a proxy for the SN site \citep{Prieto2008, Kelly2012}, adopting the
luminosity-metallicity relation for SDSS galaxies \citep{Tremonti2004}, gives
systematically larger abundances than those measured in the SNe regions \citep{Modjaz2011}.
This offset is not surprising since galaxies targeted for SNe searches
are more luminous and massive and consequently have higher abundances
and they can have abundance gradients in their disks.
% Modjaz et al. include SNe found in both targeted surveys and
% serendipitously detected SNe.  although including both targeted and non
%targeted galaxies should mitigate systematic effects.
To understand SN progenitors it is important to establish the
systematic effects, which arise from the oxygen abundance
determinations, when only an integrated spectrum can be obtained
(e.g. low-luminosity or distant, unresolved galaxies), as well as the
implications of metallicity gradients in large massive galaxies.









Over the past decade, the deep HST Treasury surveys (GOODS, CANDELS, CLASH) have all enabled "piggy- back" Type Ia SN searches, which have collectively accumulated scores of SN detections that reach to uniquely high redshifts \citep{Riess2007, Rodney2014b, Graur2014}.  
The ongoing HFF program now provides a powerful new tool for the discovery of particularly high-z SNe. What sets this survey apart is the unique depth of each visit, reaching m$_{lim,3\sigma}$(F160W) $\sim$27.9 (AB), $\sim$ 1 mag deeper than CANDELS/CLASH per epoch. Gravitational lensing in the prime fields can also magnify fluxes significantly, making it possible to detect distant background events. 
The Hubble Frontier Field survey thus provides the first opportunity to discover SNe at 2 < z < 3, building up a small but important “New Frontier” sample.
Because Type Ia SNe are standardizable candles, we can used lensed SNe Ia to directly measure the true lensing magnification $\mu$ and confront the predictions from existing lens models \citep[e.g.][]{Riehm2011, Li2012, Patel2014}. Figure 1 shows a Type Ia SN with a magnification $\mu$ = 2.00 $\pm$ 0.19 found in our HFF SN programme \citep{Rodney2015}. The magnification of this SN is systematically overestimated by all existing mass models, with some discrepant by > 5$\sigma$. Thus, our sample of lensed SNe Ia is already proving to be a very valuable tool for testing galaxy cluster dark matter models, which will be particularly valuable for the study of z > 8 galaxies magnified by these clusters \citep[e.g.][]{Zheng2012}.
The unique combination of deep imaging, strong lensing, and rapid cadence in the HFF program has also pro- vided two very exciting discoveries of multiply-imaged transients. In January and August of 2014, we observed two short transient events in separate images of the same strongly lensed galaxy, measured to be at z=1.005 with X-shooter (Figures 1 and 2). Collectively nicknamed “Spock”, both of these events are too faint to be a normal SN and too bright to be a stellar flare. The light curves are also faster than expected from a He shell explosion on a white dwarf \citep[a “.Ia” event][]{Bildsten2007}, and fainter than any of the “fast optical transients” yet seen in wide-field ground-based surveys \citep[e.g.][]{Kasliwal2010, Poznanski2010a, Vinko2014}.
Lens models (and Occam’s razor) suggest that these two events are most likely spatially coincident on the source plane. If the two events were also coincident in time, then this could be an example of an extremely rare neutron star collision \citep[a “kilonova”][]{Tanvir2013, Barnes2013}. If not, these may be two separate outbursts from an extremely bright nova with a remarkably fast recurrence timescale of $\sim$1 year. This would be a unique nova, as it would have a recurrence timescale on par with the most extreme examples known \citep{Tang2014} and would also be at least an order of magnitude more luminous than a typical nova.
In November of 2014 we discovered another exciting transient, this time with four distinct sources appearing almost simultaneously in a strongly lensed spiral galaxy at z = 1.5. Dubbed “SN Refsdal,” this is the first ever example of a strongly lensed SN with multiple resolved images \citep[][Figure 3]{Kelly2014}. The Einstein Cross configuration is generated by a galaxy-scale lens, but the SN host galaxy is also multiply imaged by the cluster, so we expect to see SN Refsdal return elsewhere in the cluster field in 1–5 years \citep{Oguri2015, Sharon2015}. Measurements of the relative magnifications and time delays among these multiple images will soon deliver an unprecedented suite of powerful new mass model constraints.



