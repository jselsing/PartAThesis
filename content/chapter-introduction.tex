% !TEX root = ../thesis-example.tex
%
\chapter{Introduction}
\label{sec:intro}

\cleanchapterquote{Astronomy? Impossible to understand and madness to investigate.}{Sophocles}{} \\

In this Part A Thesis I will summarize some of the work I have been doing the first two years of my PhD, while also officially enrolled in the masters programme. The work I have done is mainly divided into three subjects: Quasars, GRBs and supernovae and is related to various aspects of the subjects. 

For QSOs, I have researched the average properties of a sample of bright $M_{r} \leq -27.5$ QSOs and created a composite spectrum for community usage. I showed an application of the composite in inferring dust content of the quasar host galaxies. Additionally, I find a steeper slope of the quasar power-law continuum as compared to the traditionally assumed one, indicating an intrinsically harder continuum. This project therefore investigates both intrinsic properties of the quasar phenomena, but also allows for the conditions of the intervening to be inferred, both in the quasar host or potential absorption systems in the line-of-sight.

For GRBs, I am involved with the X-shooter GRB collaboration and will be investigating the average properties of the GRB optical afterglows for the sample we are building. Data-collection for the sample is continuous and we have an ESO TOO-program to do ground-based follow-up of the optical afterglow form GRBs detected with \textit{Swift}.  I have developed an algorithm to normalize the afterglows which is then used  as a product of the afterglow sample, for example for the absorption-line studies. Once a sufficient sample size is reached, I will lead a project to construct a composite afterglow spectrum. This afterglow composite can then be used twofold: as a cross-correlation template to determine redshift for future discovered noisy afterglows and as a way to explore the average properties of the optical afterglows, thereby also allowing identification of potentially interesting outliers for further observations. 

For SNe, I am involved in two projects. One related to the explosion environments of type Ic and Ic-BL supernovae without accompanying GRBs, where we have obtained a sample of 19 supernova host galaxies observed with an IFU, allowing spatially resolved diagnostics to be calculated for the stellar progenitor population, to investigate whether the explosion environments of the two types of supernovae differ. This can be used to constrain the explosion mechanisms for the different kinds of SNe. This project is in an indirect way related to an investigation of the GRB progenitor system, because fast-ejecta SNe are sometimes seen associated with GRBs and the reason why some GRBs have SNe and others don't remains a mystery. The other supernova-project I am involved with is a collaboration with the Frontier Fields program, where spectroscopic follow-up of potential high-redshift supernovae are carried out using X-shooter. The scientific rationale behind this project is the increased chance of "serendipitously" discovering high-redshift SN in the Frontier Fields because of the strong gravitational lensing the galaxy clusters targeted. If high-redshift SN are of the Ia-type, this will help constrain our current cosmology where the high-redshift SNe carry the highest leverage in terms of discerning between different world-models. As part of this work I am working on the quadruply lensed supernova "SN Refsdal" where the X-shooter observations have helped constrain the SN type.

This report will present a brief status of the work that have been done in the three fields and how it relate this to my own research. Apart from the research I have been doing, I have taken 30 ECTS points distributed in the following PhD-schools and regular courses: "Coping with the challengers of a PhD + Scientific Writing", "Responsible Conduct of Research", "Introduction to University Pedagogy", "Summer Schools in Statistics and Computation for Astronomers", "Introduction to sub-mm interferometry and science with ALMA", "Advanced Methods in statistical data analysis", "Classical Astrophysical Papers". Additionally I have taught the 840 hours including Mechanics 1 and 2 courses for which I was awarded the Jens-Martin prize for excellence in teaching.

I will conclude with an outlook of what I will be working on in the last two years of my PhD. 

\clearpage

{\Large Publications} \\

{\large Paper I} \\
\textit{An X-shooter composite of bright $1<z<2$ quasars from UV to Infrared} \\
\textbf{Selsing, J}; Fynbo, J. P. U.; Christensen, L;  Krogager, J.-K. \\
\textit{Submitted to A\&A} \\

{\large Paper 2} \\
\textit{Spectroscopic classification and confirmation of the first multiply imaged supernova}\\
P. L. Kelly, G. Brammer, \textbf{J. Selsing}, S. A. Rodney, J. Hjorth, R. J. Foley, L Christensen, and et al. \\
\textit{Advanced draft ready} \\


{\large Paper 3} \\
\textit{Spectrophotometric analysis of GRB afterglow extinction curves with X-shooter} \\
J. Japelj, S. Covino, A. Gomboc, S. D. Vergani, P. Goldoni,\textbf{ J. Selsing}, Z. Cano, V. D'Elia, H. Flores, J. P. U. Fynbo, F. Hammer, J. Hjorth, P. Jakobsson, L. Kaper, D. Kopač, T. Krühler, A. Melandri, S. Piranomonte, R. Sánchez-Ramírez, G. Tagliaferri, N. R. Tanvir, A. de Ugarte Postigo, D. Watson, R. A. M. J. Wijers \\
\citet{Japelj2015} \\

{\large Paper 4} \\
\textit{GRB hosts through cosmic time - VLT/X-shooter emission-line spectroscopy of 96 GRB-selected galaxies at 0.1 < z < 3.6} \\
T. Krühler, D. Malesani, J. P. U. Fynbo, O. E. Hartoog, J. Hjorth, P. Jakobsson, D. A. Perley, A. Rossi, P. Schady, S. Schulze, N. R. Tanvir, S. D. Vergani, K. Wiersema, P. M. J. Afonso, J. Bolmer, Z. Cano, S. Covino, V. D'Elia, A. de Ugarte Postigo, R. Filgas, M. Friis, J. F. Graham, J. Greiner, P. Goldoni, A. Gomboc, F. Hammer, J. Japelj, D. A. Kann, L. Kaper, S. Klose, A. J. Levan, G. Leloudas, B. Milvang-Jensen, A. Nicuesa Guelbenzu, E. Palazzi, E. Pian, S. Piranomonte, R. Sanchez-Ramirez, S. Savaglio, \textbf{J. Selsing}, G. Tagliaferri, P. M. Vreeswijk, D. J. Watson, D. Xu \\
\citet{Kruhler2015} \\


\clearpage

\section{Quasars: Cosmic Lighthouses}
\label{sec:intro:qso}

The word quasar is derived from quasi-stellar radio source objects which is what quasars were initially detected as: point-like radio sources without an optical counterpart. When the first optical counterpart was discovered it indeed resembled a blue star, but spectroscopic follow-up revealed a confusing pattern with very broad emission lines superposed in a blue continuum. When the lines were matched to the correct atomic species, a cosmological origin was established which posed a problem for the mechanisms powering the source because the extreme energies observed. The Eddington limit is the maximal luminosity a gravitationally bound object of a given mass can have without radiation pressure tearing the object apart and at the luminosities inferred from the cosmological origin of the quasars and the observed brightness the required mass of the source extremely high. One of the first quasars discovered, 3C 273, is estimated to have a mass of $\sim900 \times 10^{6} M_\odot$ \citep{Peterson2004} which far exceeds the maximal possible mass a star can have before it would immediately collapse to a back hole \citep{Belczynski2010}. Quasars also exhibits a large degree of variability in the observed flux on differing time-scales in the order of weeks and years with shorter time-scale variability is superposed on longer variability periods. The short time-scales observed requires a relatively small emitting regions which poses a problem for a potential stellar origin of the emission and gives merit to the idea of a black hole powering the emission.

The optical-to-ultraviolet emission mechanism of quasars are by now quite well understood \citep{Elvis1994}. A supermassive black hole at the center of a galaxy is surrounded by a hot accretion disc which emits a featureless thermal continuum \citep{Shakura1973, Pereyra2006}. The central continuum photoionizes a region of hot clouds, and further out, cold clouds which gives rise to lines with both broad and narrow components \citep{Elvis2001}.  Varying conditions in the clouds ensure that each ionic species have optimal conditions to produce line emission \citep{Baldwin1995}. Despite the apparent different conditions in which the emission arises, the overall shape of most quasar spectra look similar \citep{Dietrich2002}. The remarkable uniformity of the average spectral properties across luminosity and redshift indicate very similar underlying physical mechanisms which can be understood in terms of Eigenvector 1 (\citep{Boroson1992, Francis1992}) where the Eddington ratio drives the relative strength of the lines and orientation effects influences the observed kinematics of the lines \citep{Shen2014a} accounting for the majority of the inter-quasar variation. This means that the accretion rate and the mass of the black hole largely determines the spectral appearance of a given quasar. Quasars are a part of a more general class of objects called Active Galactic Nuclei (AGN) and the different classes of AGN are in part believed to be viewing effects \citep{Elvis2001}.


The average quasar spectrum of the sample presented in Paper I is shown in Fig. \ref{fig:intro:qsospec} where the position of the most prominent emission lines has been marked. The blue dashed line is a power-law fit to the spectral regions deemed free of contaminating emission lines and it can be seen that a pure power-law relatively well describes the continuum which is also the shape that is predicted from \citep{Pereyra2006}.

\begin{figure}[htb]
	\includegraphics[width=\textwidth]{gfx/qsospec}
	\caption{Quasar composite spectrum from Paper I. The brown line is the average quasar spectrum where several of the prominent lines have been marked. The blue dashed line is a power-law fit to the regions free of contaminating emission and  the dark green line is an average spectrum constructed for all SDSS spectra fulfilling the selection criteria imposed on the sample observed with X-shooter.}
	\label{fig:intro:qsospec}
\end{figure}


Even though the general principles behind the quasar phenomena seems to be relatively well understood, important details remains especially in terms of understanding the total contribution of the quasar feedback in regulating star formation in galaxies \citep{DiMatteo2005}, how the first quasar became so massive so rapidly after the creating of the universe \cite{Wu2015a}, how the co-evolution of the quasars and their host galaxies works \citep{Ferrarese2000}, what role quasars played in reionization \citep{Hopkins2007} and what the full sample of quasars look like \citep{Krawczyk2015}. The quasar contribution to reionization depends on the quasar luminosity function (QLF) which in turn depends on the assumed slope of the quasar SED. The QLF is the intregrated luminosity of all quasars as a function of redshift and when calculating it, a representative quasar SED model is required. The quasar optical SED is usually is modelled by a power-law where the slope of the quasar continuum is determined from the average slope og large samples of quasars \citep{Vandenberk2001, Richards2006a, Shen2011, Lusso2015} where regions free of emission lines is determined "by-hand". It is very efficient to build large samples of quasars from SDSS \citep{Paris2014}, but the relatively modest wavelength coverage of the instrument install at Apache Point Observatory \citep{Gunn2006} makes it difficult to uniquely determine a representative slope. Work has been done to extend the average quasar spectrum to higher wavelengths using other instruments \citep{Glikman2006}, but a well defined sample of quasars with spectra covering the entire range from ultra-violet to near-infrared has not been carried out. The work I have carried out in Paper I seeks to address specifically this point by observing 7 extremely bright, $M_{r} \leq -27.5$, quasars which will therefore contain negligible amounts of host galaxy contamination. Additionally the composite spectrum I have generated can be used to infer the dust content of the quasar host galaxy and potential intervening absorption systems, specifically Damped Lyman Alpha (DLA) systems. DLAs are defined as having hydrogen column densities higher than $N_{\mathrm{HI}} > 2 \times 10^{20} cm^{-2}$, determined from the equivalent width of the hydrogen absorption lines. This class of objects are believed to be self-shielding systems of gas that contain a significant fraction of the neutral hydrogen at redshift $z \sim 5$ \citep{StorrieLombardi2000}. Since these objects are only observed in absorption, our understanding of these object depend heavily on our knowledge of the illuminating object. An example of inferring the dust properties and amount is \citet{Krogager2015}, where artificially reddening a composite quasar spectrum is used to fit for the amount of dust under some assumed extinction law. The quasar composite used in that work consists of the template by \citet{Vandenberk2001} stitched together with the one by \citet{Glikman2006}, to produce a quasar composite covering the range of wavelength investigated. Paper I presents a single template that covers en entire region from a homogeneous selection and hopefully it will see a great deal of community usage. The paper, template and all the code used to generate the composite is made publicly available at \url{https://github.com/jselsing/QuasarComposite} as a way to encourage open source science.

\section{Gamma-Ray Bursts: Flashes in the Dark}
\label{sec:intro:grb}


\section{Supernovae: The fiery deaths of massive stars and their environments.}
\label{sec:intro:sn}

Some of nature's most powerful explosions, long
  GRBs and different kinds of core-collapse SN, are linked to the
  deaths of massive stars. However, neither the progenitor system nor
  the production conditions that lead to each kind of explosion in a
  massive stripped star are well understood. The different kinds of
  Supernovae (SNe) are determined by the initial mass and metallicity
  of the stellar progenitor, as well as by the metallicity-dependent
  mass loss in the stellar winds at the end phase of their evolution
  and the interaction with a sufficiently close companion star. Type
  Ic SNe (SNe Ic) are explosions from the most heavily stripped
  massive stars that have been removed of both their H and He
  layers. Long duration gamma-ray bursts (GRBs) may be the most
  extreme cases of stellar explosions, and a few cases have been
  associated with broad-lined Type Ic SN \citep{Woosley2006}, while other long GRBs
  surprisingly lacked the distinct SN signatures \citep{Fynbo2006}. The connection
  between long-duration GRBs and broad-lined SNe Ic (SNe
  Ic-bl/hypernovae) and the existence of SNe Ic-bl without observed
  GRBs raises the question of what distinguishes a GRB progenitor from
  that of an ordinary SN Ic-bl without a GRB. This question may be
  answered by observing the stellar progenitors, which give rise to
  the various types of SN explosions. However, searches for SN
  progenitors in images taken before the explosions have failed,
  because the galaxies are distant and individual stars are difficult
  to distinguish \citep{Maund2005}.

\hspace{0.4cm} An indirect method is to explore the environments at
the locations of the SN and GRBs to look for systematic trends. SN
Ic-bl without observed GRBs lie in systematically more metal rich
environments than SNe with GRBs \citep{Modjaz2008}. Integral field data of the host
of GRB\,980425/SN\,1998bw show that the location of the SN (a Type
Ic-bl) is close to a very metal poor region, but that the metallicity
variations in HII regions in the dwarf galaxy host are otherwise minor
\citep{Christensen2008}.  In comparison Fig. 1 shows the Type Ic sites to be relative
metal rich. Recently the abundances at the sites of different kinds of
core collapse SNe have been determined and shows that SNe Type Ic lie
in systematically more metal rich environments than other types of
core collapse explosions (Fig. 3) \citep{Modjaz2011, Leloudas2011, Kuncarayakti2013a, Kelly2012}.  In contrast \citet{Anderson2010} argue
that metallicity is not the dominant parameter for the SN type. The
metallicity difference may have a physical origin, as a higher
metallicity may give rise to stronger line-driven winds that remove
the outer H and He layers of the star \cite{Vink2005}, a necessary condition for
a Type Ic SNe. This argument does not explain the lower abundances at
the sites of SNe Ic-bl as well as at GRB regions, since GRB
progenitors are expected to be metal poor \citep{Woosley1993}, unless the GRB-SN
progenitors underwent a peculiar evolutionary phase marked by chemical
homogeneous mixing. %A possible solution to this dilemma is a massive
%binary progenitor system, rather than a metallicity effect.

\hspace{0.4cm}
Using integrated central oxygen abundance of the host
galaxy as a proxy for the SN site \citep{Prieto2008, Kelly2012}, adopting the
luminosity-metallicity relation for SDSS galaxies \citep{Tremonti2004}, gives
systematically larger abundances than those measured in the SNe regions \citep{Modjaz2011}.
This offset is not surprising since galaxies targeted for SNe searches
are more luminous and massive and consequently have higher abundances
and they can have abundance gradients in their disks.
% Modjaz et al. include SNe found in both targeted surveys and
% serendipitously detected SNe.  although including both targeted and non
%targeted galaxies should mitigate systematic effects.
To understand SN progenitors it is important to establish the
systematic effects, which arise from the oxygen abundance
determinations, when only an integrated spectrum can be obtained
(e.g. low-luminosity or distant, unresolved galaxies), as well as the
implications of metallicity gradients in large massive galaxies.









Over the past decade, the deep HST Treasury surveys (GOODS, CANDELS, CLASH) have all enabled "piggy- back" Type Ia SN searches, which have collectively accumulated scores of SN detections that reach to uniquely high redshifts \citep{Riess2007, Rodney2014b, Graur2014}.  
The ongoing HFF program now provides a powerful new tool for the discovery of particularly high-z SNe. What sets this survey apart is the unique depth of each visit, reaching m$_{lim,3\sigma}$(F160W) $\sim$27.9 (AB), $\sim$ 1 mag deeper than CANDELS/CLASH per epoch. Gravitational lensing in the prime fields can also magnify fluxes significantly, making it possible to detect distant background events. 
The Hubble Frontier Field survey thus provides the first opportunity to discover SNe at 2 < z < 3, building up a small but important “New Frontier” sample.
Because Type Ia SNe are standardizable candles, we can used lensed SNe Ia to directly measure the true lensing magnification $\mu$ and confront the predictions from existing lens models \citep[e.g.][]{Riehm2011, Li2012, Patel2014}. Figure 1 shows a Type Ia SN with a magnification $\mu$ = 2.00 $\pm$ 0.19 found in our HFF SN programme \citep{Rodney2015}. The magnification of this SN is systematically overestimated by all existing mass models, with some discrepant by > 5$\sigma$. Thus, our sample of lensed SNe Ia is already proving to be a very valuable tool for testing galaxy cluster dark matter models, which will be particularly valuable for the study of z > 8 galaxies magnified by these clusters \citep[e.g.][]{Zheng2012}.
The unique combination of deep imaging, strong lensing, and rapid cadence in the HFF program has also pro- vided two very exciting discoveries of multiply-imaged transients. In January and August of 2014, we observed two short transient events in separate images of the same strongly lensed galaxy, measured to be at z=1.005 with X-shooter (Figures 1 and 2). Collectively nicknamed “Spock”, both of these events are too faint to be a normal SN and too bright to be a stellar flare. The light curves are also faster than expected from a He shell explosion on a white dwarf \citep[a “.Ia” event][]{Bildsten2007}, and fainter than any of the “fast optical transients” yet seen in wide-field ground-based surveys \citep[e.g.][]{Kasliwal2010, Poznanski2010a, Vinko2014}.
Lens models (and Occam’s razor) suggest that these two events are most likely spatially coincident on the source plane. If the two events were also coincident in time, then this could be an example of an extremely rare neutron star collision \citep[a “kilonova”][]{Tanvir2013, Barnes2013}. If not, these may be two separate outbursts from an extremely bright nova with a remarkably fast recurrence timescale of $\sim$1 year. This would be a unique nova, as it would have a recurrence timescale on par with the most extreme examples known \citep{Tang2014} and would also be at least an order of magnitude more luminous than a typical nova.
In November of 2014 we discovered another exciting transient, this time with four distinct sources appearing almost simultaneously in a strongly lensed spiral galaxy at z = 1.5. Dubbed “SN Refsdal,” this is the first ever example of a strongly lensed SN with multiple resolved images \citep[][Figure 3]{Kelly2014}. The Einstein Cross configuration is generated by a galaxy-scale lens, but the SN host galaxy is also multiply imaged by the cluster, so we expect to see SN Refsdal return elsewhere in the cluster field in 1–5 years \citep{Oguri2015, Sharon2015}. Measurements of the relative magnifications and time delays among these multiple images will soon deliver an unprecedented suite of powerful new mass model constraints.



